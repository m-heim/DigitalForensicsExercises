\documentclass[10pt,a4paper]{article}
\usepackage[utf8]{inputenc}
\usepackage[english]{babel}
\usepackage{hyperref}
\usepackage{fancyvrb}
\usepackage{listings}
\usepackage{/home/m-heim/DigitalForensicsExercises/documentation/mhlm/mhlm}
\author{Maximilian Heim}
\title{Digital Forensics Exercises}
\begin{document}
\maketitle
\newpage
\tableofcontents
\newpage
\section{Exercise 1.1}
\subsection{Useful links}
Time for Truth: Forensic Analysis of NTFS Timestamps - \url{https://eprints.cs.univie.ac.at/7091/}
\subsection{a) Which timestamps do Linux and Windows provide?}
\paragraph{Linux}
\begin{enumerate}
    \item atime - Acess time
    \item mtime - Modification time
    \item ctime - Creation time
\end{enumerate}

\paragraph{Windows}
\url{https://eprints.cs.univie.ac.at/7091/1/3465481.3470016.pdf}
\begin{enumerate}
    \item Modified - Modification time
    \item Accessed - Access time
    \item Changed - Change of file metadata via the MFT entry
    \item Birth - Creation of file va the MFT entry
\end{enumerate}

\section{Exercise 1.3: Read only}
\paragraph{Exercise description}
Ein USB-Stick und ein virtuelles Laufwerk sollen beim Anschließen an einen Rechner nicht vollständig gemountet werden,
sondern im „nur-lesen“ Modus eingebunden werden. Beschreiben Sie die notwendigen Konfigurationen und erstellen Sie
jeweils ein Script um den Vorgang zu automatisieren.
\subsection{a)}
To mount a block device read in read only mode \Verb+mount -o ro <drive> /mnt+ may be used
\subsection{b)}
First automount has to be disabled via \Verb+mountvol.exe /N+. After that the media can be connected to the system. The next step is to enable the read only flag for the volume via \Verb+attributes volume set readonly+. Then the volume may be mounted.
\url{https://superuser.com/questions/213005/how-to-mount-an-ntfs-partition-read-only-in-windows}

\section{Exercise 2.1: Forensic backup}
\paragraph{Exercise description}
Erstellen Sie ein mit dem Tool dd [Howto] ein 50MB großes virtuelles Laufwerk und formatieren Sie diese mit FAT.
Verwenden Sie hierfür das Tool mkfs [Howto]. Binden Sie das Laufwerk ein und kopieren Sie anschließend verschiedene,
beliebige Dateien auf die Partition.
Erstellen Sie eine forensische Kopie der Partition - verwenden Sie hierfür das Programm dcfldd [Website][Howto].
Beachten Sie dabei, dass das Laufwerk nicht eingebunden sein darf. Der MD5-Hash-Wert der Kopie soll dabei in eine Datei
geschrieben werden. Beschreiben Sie Ihr Vorgehen und die verwendeten Befehle. Wie sieht der Befehl aus, wenn Sie das
Image in 10 MB große Dateien aufsplitten?
\paragraph{Backup partition as image}
The following command dumps the partition into an image file and saves the hash into a file

\mhlmInlineVerb{dcfldd if=imagetest.img of=imagedump.img hashlog=image.md5 hash=md5}
\paragraph{Backup partition splitted}
This command additionally splits the contents into 10 MB files

\mhlmInlineVerb{dcfldd if=imagetest.img split=10000000 of=imagedump.img hashlog=image.md5 hash=md5}

\section{Exercise 2.2: Hash validation}

\paragraph{Exercise}
Validieren Sie den Hash-Wert aus Übung 2.1, indem Sie einen Hashwert des virtuellen Laufwerkes mit dem Programm
md5sum erzeugen. Führen Sie anschließend einen automatisierten Vergleich der beiden Hash-Werte durch. Beschreiben
Sie Ihr Vorgehen und die verwendeten Befehle.

\paragraph{Generation of hash}
md5sum imagetest.img > original.md5

\paragraph{Comparison of hashes}
cmp -n 32 --ignore-initial 0:14 original.md5 imagedumphash.md5

\bibliographystyle{plain}
\bibliography{refs}

\section{Exercise 2.7: In depth analysis SK}
\subsection{a)}
\paragraph{Partition table}
\Verb+mmls uebung_2-7.dd+
\begin{lstlisting}
    DOS Partition Table
    Offset Sector: 0
    Units are in 512-byte sectors
    
          Slot      Start        End          Length       Description
    000:  Meta      0000000000   0000000000   0000000001   Primary Table (#0)
    001:  -------   0000000000   0000002047   0000002048   Unallocated
    002:  000:000   0000002048   0000053247   0000051200   Linux (0x83)
    003:  -------   0000053248   0000104447   0000051200   Unallocated
    004:  000:002   0000104448   0000155647   0000051200   Win95 FAT32 (0x0b)
    005:  Meta      0000155648   0000204799   0000049152   DOS Extended (0x05)
    006:  Meta      0000155648   0000155648   0000000001   Extended Table (#1)
    007:  -------   0000155648   0000157695   0000002048   Unallocated
    008:  001:000   0000157696   0000204799   0000047104   Linux (0x83)
\end{lstlisting}
\subsection{b)}
\paragraph{Partition 2}
\Verb+fsstat -f ext3 -o 2048 uebung_2-7.dd+
\begin{lstlisting}
    FILE SYSTEM INFORMATION
--------------------------------------------
File System Type: Ext3
Volume Name: 
Volume ID: 627cea8be986a5a3b94e761f598eab5a

Last Written at: 2012-03-12 13:40:49 (CET)
Last Checked at: 2012-03-21 15:01:53 (CET)

Last Mounted at: 2012-03-24 14:33:01 (CET)
Unmounted properly

Source OS: Linux
Dynamic Structure
Compat Features: Journal, Ext Attributes, Resize Inode, Dir Index
InCompat Features: Filetype, 
Read Only Compat Features: Sparse Super, 

Journal ID: 00
Journal Inode: 8

METADATA INFORMATION
--------------------------------------------
Inode Range: 1 - 6401
Root Directory: 2
Free Inodes: 6379

CONTENT INFORMATION
--------------------------------------------
Block Range: 0 - 25599
Block Size: 1024
Reserved Blocks Before Block Groups: 1
Free Blocks: 12343

BLOCK GROUP INFORMATION
--------------------------------------------
Number of Block Groups: 4
Inodes per group: 1600
Blocks per group: 8192

Group: 0:
  Inode Range: 1 - 1600
  Block Range: 1 - 8192
  Layout:
    Super Block: 1 - 1
    Group Descriptor Table: 2 - 2
    Data bitmap: 102 - 102
    Inode bitmap: 103 - 103
    Inode Table: 104 - 303
    Data Blocks: 304 - 8192
  Free Inodes: 1579 (98%)
  Free Blocks: 707 (8%)
  Total Directories: 2

Group: 1:
  Inode Range: 1601 - 3200
  Block Range: 8193 - 16384
  Layout:
    Super Block: 8193 - 8193
    Group Descriptor Table: 8194 - 8194
    Data bitmap: 8294 - 8294
    Inode bitmap: 8295 - 8295
    Inode Table: 8296 - 8495
    Data Blocks: 8496 - 16384
  Free Inodes: 1600 (100%)
  Free Blocks: 3955 (48%)
  Total Directories: 0

Group: 2:
  Inode Range: 3201 - 4800
  Block Range: 16385 - 24576
  Layout:
    Data bitmap: 16385 - 16385
    Inode bitmap: 16386 - 16386
    Inode Table: 16387 - 16586
    Data Blocks: 16387 - 16386, 16587 - 24576
  Free Inodes: 1600 (100%)
  Free Blocks: 6961 (84%)
  Total Directories: 0

Group: 3:
  Inode Range: 4801 - 6400
  Block Range: 24577 - 25599
  Layout:
    Super Block: 24577 - 24577
    Group Descriptor Table: 24578 - 24578
    Data bitmap: 24678 - 24678
    Inode bitmap: 24679 - 24679
    Inode Table: 24680 - 24879
    Data Blocks: 24880 - 25599
  Free Inodes: 1600 (100%)
  Free Blocks: 720 (70%)
  Total Directories: 0
\end{lstlisting}

\paragraph{Partition 4}
\Verb+fsstat -f fat -o 104448 uebung_2-7.dd+
\begin{lstlisting}
    FILE SYSTEM INFORMATION
--------------------------------------------
File System Type: FAT16

OEM Name: mkdosfs
Volume ID: 0x38ba908
Volume Label (Boot Sector):            
Volume Label (Root Directory):
File System Type Label: FAT16   

Sectors before file system: 0

File System Layout (in sectors)
Total Range: 0 - 51199
* Reserved: 0 - 3
** Boot Sector: 0
* FAT 0: 4 - 55
* FAT 1: 56 - 107
* Data Area: 108 - 51199
** Root Directory: 108 - 139
** Cluster Area: 140 - 51199

METADATA INFORMATION
--------------------------------------------
Range: 2 - 817478
Root Directory: 2

CONTENT INFORMATION
--------------------------------------------
Sector Size: 512
Cluster Size: 2048
Total Cluster Range: 2 - 12766

FAT CONTENTS (in sectors)
--------------------------------------------
144-11287 (11144) -> EOF
11288-17183 (5896) -> EOF
17184-20971 (3788) -> EOF
20980-20983 (4) -> EOF
20984-21687 (704) -> EOF
21688-22391 (704) -> EOF
22392-23099 (708) -> EOF
23100-23219 (120) -> EOF
23220-23339 (120) -> EOF
23340-23459 (120) -> EOF
\end{lstlisting}

\paragraph{Partition 8}
\Verb+fsstat -f ext -o 157696 uebung_2-7.dd+
\begin{lstlisting}
    FILE SYSTEM INFORMATION
--------------------------------------------
File System Type: Ext4
Volume Name: 
Volume ID: d787b67ed5e90caa3f4a161a87787e76

Last Written at: 2012-03-12 13:40:49 (CET)
Last Checked at: 2012-03-21 15:03:05 (CET)

Last Mounted at: 2012-03-07 14:48:20 (CET)
Unmounted properly
Last mounted on: /home/cmoch/ueb_albsig/ext4

Source OS: Linux
Dynamic Structure
Compat Features: Journal, Ext Attributes, Resize Inode, Dir Index
InCompat Features: Filetype, Extents, Flexible Block Groups, 
Read Only Compat Features: Sparse Super, Huge File, Extra Inode Size

Journal ID: 00
Journal Inode: 8

METADATA INFORMATION
--------------------------------------------
Inode Range: 1 - 5905
Root Directory: 2
Free Inodes: 5883
Inode Size: 128

CONTENT INFORMATION
--------------------------------------------
Block Groups Per Flex Group: 16
Block Range: 0 - 23551
Block Size: 1024
Reserved Blocks Before Block Groups: 1
Free Blocks: 12620

BLOCK GROUP INFORMATION
--------------------------------------------
Number of Block Groups: 3
Inodes per group: 1968
Blocks per group: 8192

Group: 0:
  Inode Range: 1 - 1968
  Block Range: 1 - 8192
  Layout:
    Super Block: 1 - 1
    Group Descriptor Table: 2 - 2
    Group Descriptor Growth Blocks: 3 - 93
    Data bitmap: 94 - 94
    Inode bitmap: 110 - 110
    Inode Table: 126 - 371
    Data Blocks: 372 - 8192
  Free Inodes: 1947 (98%)
  Free Blocks: 7341 (89%)
  Total Directories: 2

Group: 1:
  Inode Range: 1969 - 3936
  Block Range: 8193 - 16384
  Layout:
    Super Block: 8193 - 8193
    Group Descriptor Table: 8194 - 8194
    Group Descriptor Growth Blocks: 8195 - 8285
    Data bitmap: 95 - 95
    Inode bitmap: 111 - 111
    Inode Table: 372 - 617
    Data Blocks: 618 - 16384
  Free Inodes: 1968 (100%)
  Free Blocks: 358 (4%)
  Total Directories: 0

Group: 2:
  Inode Range: 3937 - 5904
  Block Range: 16385 - 23551
  Layout:
    Data bitmap: 96 - 96
    Inode bitmap: 112 - 112
    Inode Table: 618 - 863
    Data Blocks: 864 - 23551
  Free Inodes: 1968 (100%)
  Free Blocks: 4921 (68%)
  Total Directories: 0
\end{lstlisting}
\subsection{c)}
\paragraph{Partition 2}
\Verb+fls -o 2048 -f ext uebung_2-7.dd+
\begin{lstlisting}
d/d 11: lost+found
r/r 12: ebc36c92-3886-11e1-af06-5c260a3d892a.mp3
r/r 13: ebc89640-3886-11e1-af06-5c260a3d892a.mp3
r/r 14: 4cafbc82-389a-11e1-af06-5c260a3d892a.mp3
r/r 15: a5df3984-38bf-11e1-af06-5c260a3d892a.txt
r/r * 16: 7e070e10-38ff-11e1-af06-5c260a3d892a.txt
r/r * 17: 64349836-3936-11e1-af06-5c260a3d892a.txt
r/r 18: 2f2e3c1c-3942-11e1-af06-5c260a3d892a.jpg
r/r 19: d0b8aa62-3948-11e1-af06-5c260a3d892a.jpg
r/r 20: 4f20572c-395f-11e1-af06-5c260a3d892a.bmp
r/r 21: ded49934-397b-11e1-af06-5c260a3d892a.txt
r/r 22: e8ca3836-39bc-11e1-af06-5c260a3d892a.txt
r/r 23: cb04b604-39cc-11e1-af06-5c260a3d892a.txt
V/V 6401: $OrphanFiles
\end{lstlisting}
\paragraph{Partition 4}
\Verb+fls -o 104448 -f fat uebung_2-7.dd+
\begin{lstlisting}
r/r 7: d1f7f3b0-6891-11e1-af06-5c260a3d892a.mp3
r/r 12: d1faea98-6891-11e1-af06-5c260a3d892a.mp3
r/r 17: cb8b76b8-68d3-11e1-af06-5c260a3d892a.mp3
r/r * 22: 7dc51cc0-68da-11e1-af06-5c260a3d892a.txt
r/r * 27: 3d211488-68eb-11e1-af06-5c260a3d892a.txt
r/r 32: 8b103ba4-6938-11e1-af06-5c260a3d892a.txt
r/r 37: 18fb26fe-6957-11e1-af06-5c260a3d892a.bmp
r/r 42: 3cd504f6-6983-11e1-af06-5c260a3d892a.bmp
r/r 47: 9cf197c2-69cf-11e1-af06-5c260a3d892a.bmp
r/r 52: 237896fa-6a0d-11e1-af06-5c260a3d892a.txt
r/r 57: 3d86f27e-6a50-11e1-af06-5c260a3d892a.txt
r/r 62: 4df1a002-6a5d-11e1-af06-5c260a3d892a.txt
v/v 817475: $MBR
v/v 817476: $FAT1
v/v 817477: $FAT2
V/V 817478: $OrphanFiles
\end{lstlisting}
\paragraph{Partition 8}
\Verb+fls -o 157696 -f ext uebung_2-7.dd+
\begin{lstlisting}
d/d 11: lost+found
r/r 12: 77336f00-6b38-11e1-af06-5c260a3d892a.mp3
r/r 13: 7735ba62-6b38-11e1-af06-5c260a3d892a.mp3
r/r 14: 5562411e-6b86-11e1-af06-5c260a3d892a.mp3
r/r 15: 24012126-6bb8-11e1-af06-5c260a3d892a.txt
r/r * 16: 4e249608-6bf5-11e1-af06-5c260a3d892a.txt
r/r * 17: 876a0370-6c0f-11e1-af06-5c260a3d892a.txt
r/r 18: 2c430cc4-6c25-11e1-af06-5c260a3d892a.bmp
r/r 19: 6e8056fa-6c4d-11e1-af06-5c260a3d892a.jpg
r/r 20: 8ae5a6c6-6c78-11e1-af06-5c260a3d892a.gif
r/r 21: 901b0e08-6cc1-11e1-af06-5c260a3d892a.txt
r/r 22: e26033d4-6cc9-11e1-af06-5c260a3d892a.txt
r/r 23: 4e499582-6cd8-11e1-af06-5c260a3d892a.txt
V/V 5905: $OrphanFiles
\end{lstlisting}
\end{document}