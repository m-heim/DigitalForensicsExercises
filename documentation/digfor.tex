\documentclass[10pt,a4paper]{article}
\usepackage[utf8]{inputenc}
\usepackage[english]{babel}
\usepackage{amsmath}
\usepackage{amsfonts}
\usepackage{amssymb}
\usepackage{hyperref}
\usepackage{fancyvrb}
\author{Maximilian Heim}
\title{Digital Forensics Exercises}
\begin{document}
\maketitle
\newpage
\tableofcontents
\newpage
\section{Exercise 1.1}
\subsection{Useful links}
Time for Truth: Forensic Analysis of NTFS Timestamps - \url{https://eprints.cs.univie.ac.at/7091/}
\subsection{a) Which timestamps do Linux and Windows provide?}
\paragraph{Linux}
\begin{enumerate}
    \item atime - Acess time
    \item mtime - Modification time
    \item ctime - Creation time
\end{enumerate}

\paragraph{Windows}
\url{https://eprints.cs.univie.ac.at/7091/1/3465481.3470016.pdf}
\begin{enumerate}
    \item Modified - Modification time
    \item Accessed - Access time
    \item Changed - Change of file metadata via the MFT entry
    \item Birth - Creation of file va the MFT entry
\end{enumerate}

\section{Exercise 1.3: Read only}
\paragraph{Exercise description}
Ein USB-Stick und ein virtuelles Laufwerk sollen beim Anschließen an einen Rechner nicht vollständig gemountet werden,
sondern im „nur-lesen“ Modus eingebunden werden. Beschreiben Sie die notwendigen Konfigurationen und erstellen Sie
jeweils ein Script um den Vorgang zu automatisieren.
\subsection{a)}
To mount a block device read in read only mode \Verb+mount -o ro <drive> /mnt+ may be used
\subsection{b)}
First automount has to be disabled via \Verb+mountvol.exe /N+. After that the media can be connected to the system. The next step is to enable the read only flag for the volume via \Verb+attributes volume set readonly+. Then the volume may be mounted.
\url{https://superuser.com/questions/213005/how-to-mount-an-ntfs-partition-read-only-in-windows}


\bibliographystyle{plain}
\bibliography{refs}
\end{document}